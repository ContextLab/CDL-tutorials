\documentclass{beamer}
%
% Choose how your presentation looks.
%
% For more themes, color themes and font themes, see:
% http://deic.uab.es/~iblanes/beamer_gallery/index_by_theme.html
%
\mode<presentation>
{
  \usetheme{default}      % or try Darmstadt, Madrid, Warsaw, ...
  \usecolortheme{default} % or try albatross, beaver, crane, ...
  \usefonttheme{default}  % or try serif, structurebold, ...
  \setbeamertemplate{navigation symbols}{}
  \setbeamertemplate{caption}[numbered]
} 


\usepackage[english]{babel}
\usepackage[utf8x]{inputenc}
\usepackage{listings}
\title[PANDAS Tutorial Presentation]{PANDAS Tutorial Presentation}
\author{Emily}


\begin{document}

\begin{frame}
  \titlepage
\end{frame}

% Uncomment these lines for an automatically generated outline.
%\begin{frame}{Outline}
%  \tableofcontents
%\end{frame}

\section{Introduction}

\begin{frame}{Introduction}

\begin{itemize}
  \item This tutorial will guide you through the basics of PANDAS, a package that works in tandem with num.py and python
  \item After completing this tutorial you should have a better understanding of the foundations of PANDAS
  
\end{itemize}


\end{frame}

\section{Some \LaTeX{} Examples}

\subsection{How to use the template tutorial}

\begin{frame}{How to use this tutorial}

\begin{itemize}
  \item Before starting the tutorial you should pip install the requirements.txt a
  \item To get the most out of this tutorial start with the README, then these slides, then whichever notebooks are most relevant to your needs. 
  \item One could either do the individual tutorials and walkthrough first, or if you are looking to solve a specific problem with  PANDAS, look to see if there is a notebook on that issue. 
\end{itemize}

\end{frame}

\begin{frame}{Graph}

% Commands to include a figure:
%need to go to Project-->files --> upload a new file can't upload from desktop or docs
\begin{figure}
\includegraphics[width=0.5\textwidth]{DF_example.png} 
\caption{\label{fig:your-figure} Above is an example of a dataset being analyzed using PANDAS' describe feature.}
\end{figure}

\end{frame}{}

\subsection{Code}

\begin{frame}[fragile]{Code}

Here is an example of code from the "Rolling Apply" notebook. Rolling apply is especially useful as it applies a function across a DataFrame or a series.  

\begin{verbatim}
def times_3 (x):
    if x >4:
        return x * 3
    else:
        return x)
df3['A'].apply(times_3)

\end{verbatim}

\end{frame}

\subsection{Summary}

\begin{frame}{Summary}

\begin{itemize}
\item PANDAS can be used in the lab for manipulating data in many different ways   
\item It has particular strengths in organizing and visualizing large sets of data 
\item If you want to learn more about PANDAS here are some useful places to look:
	\begin{itemize}
    \item https://pandas.pydata.org
    \item https://www.datacamp.com/community/blog/python-pandas-cheat-sheet
	\end{itemize}
\end{itemize}

\end{frame}

\end{document}

%\How to make a table
%\begin{table}
%\centering
%\begin{tabular}{l|r}
%Item & Quantity \\\hline
%Widgets & 42 \\
%Gadgets & 13
%\end{tabular}
%\caption{\label{tab:widgets}An example table.}
%\end{table}



